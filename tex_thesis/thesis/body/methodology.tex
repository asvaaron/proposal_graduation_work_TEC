


\chapter{Methodology}
\label{chapter:methodology}
\epigraph{``It doesn't matter how beautiful your theory is, it doesn't matter how smart you are. If it doesn't agree with experiment, it's wrong.''}{\vspace{10pt}Richard Feynman }

\newpage

Presents a brief introduction to kind of statistical tool used in the experiment and the details of the experiment of the research. 
\section{Experiment Design}
\label{section:experiment-design}

The experiment will consist of 8 unique configurations derived from the combination of the three factors and their respective levels $(2^k, \text{with K} = 3 = 8)$. For each configuration, the POSTER Visual Transformer will be trained on a facial expression dataset RAF-DB \ref{sub:raf-db}, and the performance will be evaluated based on a chosen metric, such as classification accuracy \ref{sub:accurracy} and F1-score \ref{sub:f1-score}.

The models will be trained in a controlled environment, where the only variations between runs are the levels of the factors being tested. This will allow for a rigorous comparison of how pretrained weights, model size, learning rate and MLP head modifications influence the transformer’s performance in FER tasks.


\section{Factors and Levels (if needed):}

\section{Measures and Combinations of the Experiment (if needed)}
The example \textbf{Table \ref{tab:table}} summarizes the factors and levels used in the experiments.

\begin{table}[H]
\centering
\caption{Factors and Levels}
\label{tab:factors_and_levels}
\begin{tabular}{clll}
\hline
\multicolumn{1}{l}{}             & \multicolumn{3}{c}{\textbf{Factors}}                       \\
\multicolumn{1}{l}{}             & \textbf{Factor 1}  & \textbf{Factor 2} & \textbf{Factor 3} \\ \hline
\multirow{3}{*}{\textbf{Levels}} & Pretrained Weights & Model Size        & MLP Head          \\
                                 & Yes                & Base              & No Changes        \\
                                 & No                 & Large             & \textbf{Modified MLP}      \\ \hline
\end{tabular}
\end{table}


%\begin{table}[h]
%	\caption{Example table.}
%	\label{tab:table}
%	\begin{center}
%		\begin{tabular}{|l|l|l|l|l|}
%			\cline{2-5}
%			\multicolumn{1}{c|}{}	& \multicolumn{4}{c|}{Factor} \\ 
%			\cline{2-5}
%			\multicolumn{1}{c|}{}	& Factor 1 & Factor 2 & Factor 3 & Factor 4 \\ \hline
%			Levels
%			& Level 1 & Level 1	& Level 1  & Level 1   \\
%			& Level 2 & Level 2  & Level 2 & Level 2 \\
%			& Level 3 & Level 3 & Level 3 & Level 3 \\
%			& Level 4 &    & Level 4  &   \\ 
%			&  &    & Level 5  &   \\ 
%			&  &    & Level 6  &     \\\hline
%		\end{tabular}
%	\end{center}
%\end{table}



\section{Response Variable}
Description of the response variable used in the research.

\section{Description of the set up of the experiments performed}

\section{Description of the inputs of the system}

\section{Data Recollection}

Description of the data recollection process. Usually, it is a good idea to present a pointer to a GitHub page where all data and scripts used are available.

\section{Statistical Tool Used}

Include a section briefly explaining the statistical tool used to analyze the data from the experiments.
\subsection{Hypothesis of the statistical tool used}

\textbf{Null hypothesis:}

\textbf{Alternate hypothesis:} 
