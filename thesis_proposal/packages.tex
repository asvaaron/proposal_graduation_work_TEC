%For spanish we need indentation on the first line
%\usepackage{indentfirst}

%math package
\usepackage{amsthm}
\usepackage{amsmath}
\usepackage{amssymb}
\usepackage{pseudocode}

\newcommand{\lstnumberautorefname}{Listing}


\newtheorem{definition}{Definition}[subsection]
\newcommand{\definitionautorefname}{Definition}

\newtheorem{theorem}{Theorem}[subsection]
\newtheoremstyle{theorem}{}{}{\itshape}{}{\bfseries}{.}{.5em}{\thmnote{#3's }#1}
\newcommand{\theoremautorefname}{Theorem}

\newtheorem{lemma}{Lemma}[subsection]
\newtheoremstyle{lemma}{}{}{\itshape}{}{\bfseries}{.}{.5em}{\thmnote{#3's }#1}
\newcommand{\lemmaautorefname}{Lemma}

\newtheorem*{remark}{Remark}
\newcommand{\remarkautorefname}{Remark}

\newcommand{\algorithmautorefname}{Algorithm}

\usepackage{listings}

\lstdefinestyle{customc}{
  language=C++,
  belowcaptionskip=1\baselineskip,
  breaklines=true,
  frame=l,
  xleftmargin=\parindent,
  showstringspaces=false,
  basicstyle=\linespread{0.9}\scriptsize,
  emptylines=0,
  keywordstyle=\scriptsize\ttfamily\bfseries\color{blue!40!black},
  commentstyle=\mdseries\scriptsize\color{gray!40!black},
  identifierstyle=\scriptsize\ttfamily\color{darkgray!15!black},
  stringstyle=\scriptsize\ttfamily\color{green!40!black},
}

\lstset{style=customc}

\usepackage{textcomp}
\DeclareUnicodeCharacter{8208}{fi}

\usepackage{supertabular}

%tables
\usepackage{multirow}
\usepackage{booktabs}
%\usepackage[table]{xcolor}

%figures in the exact position as code
\usepackage{float}

% Prevent latex from expanding to fill page
\raggedbottom

%Use the margins requested
\usepackage[hmargin={3.5cm,2.5cm},vmargin=2.5cm]{geometry}

%bold font for caption
\usepackage[labelfont=bf]{caption}

%\usepackage{digsig}

%Improved bibliography
\usepackage[round,sort,numbers,authoryear]{natbib}
\usepackage{usebib}
\bibpunct{[}{]}{;}{n}{,}{,}

% To define spacing
\usepackage{setspace}

% Bible references
\usepackage{verse}
\usepackage{bibleref}

% We use these packages for making the nice logo on the title page
\usepackage[pdftex]{graphicx}

%chemical
\usepackage{chemformula}

% Use input characters instead of scape codes
\usepackage[utf8x]{inputenc}

% Generate fancy chapter titles
\usepackage[Sonny]{fncychap}
\ChNameAsIs
\ChNumVar{\Large}
\ChNameVar{\fontsize{50}{60}}

%no word breaking
%\usepackage[none]{hyphenat}
\tolerance=1
\emergencystretch=\maxdimen
\hyphenpenalty=10000
\hbadness=10000

%Use Fancy headers
\usepackage{fancyhdr}
%\pagestyle{fancy}
%\fancyhead[LO]{\nouppercase{\leftmark}}
\lhead{\nouppercase{\rightmark}}

% Generate pretty PDF with links on the TOC
\usepackage{mdwlist}
\usepackage{alltt}
\usepackage[bookmarks=true,linktoc=all, colorlinks=true, pdftitle={\thesistitle}, pdfauthor={\pdfauthor}, pdfsubject={\thesistitle}, pdfkeywords={\thesiskeywords},
linkcolor=darkgray,filecolor=darkgray,urlcolor=darkgray,citecolor=darkgray]{hyperref}


\addto\extrasenglish{%
  \renewcommand{\chapterautorefname}{Chapter}%
  \renewcommand{\sectionautorefname}{Section}%
  \renewcommand{\subsectionautorefname}{Section}%
  \renewcommand{\subsubsectionautorefname}{Section}%
}

\newcommand*{\fullref}[1]{\hyperref[{#1}]{\autoref*{#1}, \nameref*{#1}}} % One single link

\newcommand*{\listingref}[1]{\hyperref[{#1}]{\autoref*{#1} \nameref*{#1} [page~\pageref{#1}]}}

% Allow epigraphs at the beginning of chapters
\usepackage{epigraph}
\setlength{\epigraphrule}{0pt}
\setlength{\afterepigraphskip}{10pt}

% Define abstract environment since it doesn't exists on book class
\newenvironment{abstract}%
{\cleardoublepage\null \vfill \begin{center}%
\bfseries \abstractname \end{center}}
{\vfill\null}

% Generation of nomenclature
\usepackage{nomencl}
\makenomenclature

% Generation of acronyms
\usepackage[footnote,withpage,printonlyused]{acronym}

% For color definitions
\usepackage{color}

% Comment this out when is not a draft
%\usepackage{draftwatermark}
%\SetWatermarkLightness{0.95}
%\SetWatermarkFontSize{5cm}
%\SetWatermarkScale{5}
%\SetWatermarkText{DRAFT}

% Custom commands
\newcommand{\HRule}{\rule{\linewidth}{0.5mm}}

\usepackage{cclicenses}

\usepackage{soulutf8}
\usepackage{varwidth}

\definecolor{light-gray}{gray}{0.75}

\newcommand\backBox[1]{%
  \colorbox{light-gray}{\begin{varwidth}{\dimexpr\linewidth-2\fboxsep}#1\end{varwidth}}}
  
\newcommand\hightligher[1]{%
  \colorbox{yellow}{\begin{varwidth}{\dimexpr\linewidth-2\fboxsep}#1\end{varwidth}}}
  
  
\usepackage{tcolorbox}
\usepackage{tabularx}
\newcolumntype{L}{>{\raggedright\arraybackslash}X}
\usepackage{array}
\usepackage{colortbl}

\newcolumntype{Y}{>{\raggedleft\arraybackslash}X}

\tcbset{tab1/.style={fonttitle=\bfseries\large,fontupper=\normalsize\sffamily,
colback=white!10!white,colframe=red!75!black,colbacktitle=white!40!white,
coltitle=black,center title,freelance,frame code={
\foreach \n in {north east,north west,south east,south west}
{\path [fill=gray!75!black] (interior.\n) circle (3mm); };},}}

\tcbset{tab2/.style={fonttitle=\bfseries,fontupper=\normalsize\sffamily,
colback=white!10!white,colframe=gray!50!black,colbacktitle=white!40!white,
coltitle=black,center title}}

\usepackage{pgfgantt}
\usepackage{soul}
%\include{eshyph.tex}
\usepackage{tikz}
\usepackage{verbatim}
\usetikzlibrary{calc,trees,positioning,arrows,chains,shapes.geometric,%
    decorations.pathreplacing,decorations.pathmorphing,shapes,%
    matrix,shapes.symbols}

\tikzset{
>=stealth',
  punktchain/.style={
    rectangle, 
    rounded corners, 
    % fill=black!10,
    draw=black, very thick,
    text width=10em, 
    minimum height=3em, 
    text centered, 
    on chain},
  line/.style={draw, thick, <-},
  element/.style={
    tape,
    top color=white,
    bottom color=blue!50!black!60!,
    minimum width=8em,
    draw=blue!40!black!90, very thick,
    text width=10em, 
    minimum height=3.5em, 
    text centered, 
    on chain},
  every join/.style={->, thick,shorten >=1pt},
  decoration={brace},
  tuborg/.style={decorate},
  tubnode/.style={midway, right=2pt},
}

\usepackage{pdflscape}
\usepackage{svg}

% TODO List support

\usepackage{xargs}                      % Use more than one optional parameter in a new commands

\usepackage{algorithm}
\usepackage[noend]{algpseudocode}


\usepackage[colorinlistoftodos,prependcaption,textsize=tiny]{todonotes}
\newcommandx{\unsure}[2][1=]{\todo[linecolor=red,backgroundcolor=red!25,bordercolor=red,#1]{#2}}
\newcommandx{\change}[2][1=]{\todo[linecolor=blue,backgroundcolor=blue!25,bordercolor=blue,#1]{#2}}
\newcommandx{\addsummary}[2][1=]{\todo[linecolor=yellow,backgroundcolor=yellow!25,bordercolor=yellow,#1]{Add summary of: #2}}

\newcommandx{\addmoreinfo}[2][1=]{\todo[linecolor=yellow,backgroundcolor=yellow!25,bordercolor=yellow,#1]{Add more information about: #2}}

\newcommandx{\rewritethis}[2][1=]{\todo[linecolor=pink,backgroundcolor=pink!25,bordercolor=pink,#1]{Rewrite: #2}}

\newcommandx{\improvement}[2][1=]{\todo[linecolor=green,backgroundcolor=green!25,bordercolor=green,#1]{#2}}
\newcommandx{\thiswillnotshow}[2][1=]{\todo[disable,#1]{#2}}


\newcommand{\presentationitem}[1]{%
\iftoggle{paper}
{
  \subsubsection{#1}
}
{
  \item
  \def\@currentlabelname{#1}%
  \textbf{#1}
}
}

\newcommand{\presentationitembig}[1]{%
\iftoggle{paper}
{
  \subsection{#1}%
}%else
{%
  \item%
  \textbf{#1:}%
  \def\@currentlabelname{\emph{#1}}%
}
}


\newenvironment{wbstaks}
{
}
{
}

\newcommand{\presentationbreak}{}

\usetikzlibrary{decorations.text,calc,arrows.meta}


\usetikzlibrary{lindenmayersystems,arrows.meta}
\newcount\quadrant
\pgfdeclarelindenmayersystem{cayley}{
  \rule{A -> B [ R [A] [+A] [-A] ]}
  \symbol{R}{ \pgflsystemstep=0.5\pgflsystemstep } 
  \symbol{-}{
    \pgfmathsetcount\quadrant{Mod(\quadrant+1,4)}
    \tikzset{rotate=90}
  }
  \symbol{+}{
    \pgfmathsetcount\quadrant{Mod(\quadrant-1,4)}
    \tikzset{rotate=-90}
  }
  \symbol{B}{
    \draw [dot-cayley] (0,0) -- (\pgflsystemstep,0) 
       node [font=\footnotesize, midway, 
         anchor={270-mod(\the\quadrant,2)*90}, inner sep=.5ex] 
           {\ifcase\quadrant$a$\or$b$\or$c$\or$d$\fi};
    \tikzset{xshift=\pgflsystemstep}
  }
}
\tikzset{
  dot/.tip={Circle[sep=-1.5pt,length=3pt]}, cayley/.tip={Stealth[]dot[]}
}

%Glossary
\usepackage[acronym]{glossaries}

\usepackage{csvsimple}

% Remove this
\usepackage{todonotes}
