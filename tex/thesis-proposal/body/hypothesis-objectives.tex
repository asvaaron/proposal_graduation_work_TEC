\chapter{Hypothesis and Objectives}
\label{chapter:hypothesis-objectives}

This chapter presents the hypothesis formulated for the thesis. Also, it displays the established objectives with its corresponding deliverables. Last section gives a summary of the scope and limitations defined for this research.

\section{Hypothesis}
\label{section:hypothesis}

The hypothesis for this research effort reads as follows:

%``Our proposed method $P$ in the research can improve by $Q$\footnote{If numbers are involved, usually an explanation of why that specific number was used is given in a footnote \cite{examplereference2}.} the relation in metric 1 and metric 2\footnote{The \cite{examplereference} definition of is used.}.''


''By systematically exploring and evaluating various modifications to the Multi-Layer Perceptron (MLP) head architecture of a Visual Transformer for Facial Expression Recognition (FER), it is possible to surpass the current state-of-the-art performance achieved by the POSTER model on the RAF-DB dataset. Specifically, this study aims to improve the accuracy from POSTER's reported 92.05\% to at least 93.00\%.''


\section{Objectives}
\label{section:objectives}
The current research established the following objectives:

\subsection{Main Objective}
\label{section:main-objective}
\label{objective:main}
%Main Objective Text.

To enhance the performance of Visual Transformers in Facial Expression Recognition by introducing novel modifications to the MLP head architecture and evaluating their effectiveness on an In-the-Wild Real World Affective Faces Dataset(RAF-DB).

\subsection{Specific Objectives}
\label{section:specific-objectives}

\begin{enumerate}
	\item Evaluate how the proposed modifications to the MLP head architecture provides insightful attention maps or feature representations that explain the model's decision-making process in FER.
	
\item Investigate the modified Visual Transformer model to adversarial image environments, noise, or data perturbations (e.g., occlusions, lighting variations, or slight facial deformations) or human fairness such as demographic biases (age, gender, ethnicity) using RAF-DB as a in-the-wild challenging dataset. 
	
\item Explore how the new MLP head architectures improve the model’s resilience to a in-the-wild dataset by employing standard FER evaluation metrics (e.g., accuracy, F1-score) as well as metrics that specifically address in-the-wild challenges for example head pose-invariant accuracy \cite{guo_occrob_2023} or occlusion robustness metrics like for example  Mean Average Precision (MAP) \cite{marcu_pitfalls_2022} \cite{ranjan_light-weight_2018}.
	
\end{enumerate}

\subsection{Deliverables}

This section provides the deliverables assigned for each specific objective defined for the thesis.


% Please add the following required packages to your document preamble:
% \usepackage{booktabs}

\renewcommand{\arraystretch}{2.5}
\begin{table}[H]
\label{tb:table_deliverable}
\caption{Deliverables for the Research}
\begin{tabular}{@{}llp{10cm}@{}}
\toprule
\textbf{ID} & \textbf{Name} & \textbf{Description} \\ \midrule
COD-01 & Modified POSTER & 
\parbox[t]{10cm}{Modified version of the POSTER ViT with changes in Head architecture} \\
COD-02 & POSTER Training Script & 
\parbox[t]{10cm}{Script for training the modified POSTER ViT with the RAFDB database} \\
COD-03 & Experiment Script & 
\parbox[t]{10cm}{A script to automate the data collection, experimental runs, and hypothesis testing} \\
DOC-01 & Statistical Report & 
\parbox[t]{10cm}{Report with details about results and conclusions obtained after conducting statistical analysis} \\
DOC-02 & Article Draft & 
\parbox[t]{10cm}{A draft of a scientific article with the research, ViT changes, findings, and conclusions} \\
DOC-03 & Final Thesis Document & 
\parbox[t]{10cm}{A final thesis document with all the previous deliverables' details, including an in-depth academic scientific analysis relevant to this research} \\
PRE-01 & Thesis Proposal Presentation & 
\parbox[t]{10cm}{A presentation encapsulating the scope of the study and a summary of the proposal} \\
PRE-02 & Thesis Defense Presentation & 
\parbox[t]{10cm}{A presentation with all the materials required for the thesis defense} \\ 
\bottomrule
\end{tabular}
\end{table}

\section{Scope and Limitations}

This section includes the scope of the investigation:

%Should include the scope and limitations of the developed research.

\begin{itemize}
\item This research focuses exclusively on predicting 7 discrete class FER emotions —happy, sad, surprise, fear, disgust, anger, and neutral. It deliberately excludes other FER prediction approaches such as analyzing Facial Action Units (AUs) or the Valence-Arousal (VA) model, which involve a more nuanced or continuous representation of emotion. \cite{mollahosseini_affectnet_2019}\cite{kollias_affect_2021}.

\item The project will not include research that aims to analyze emotional states from dynamic sequences (video-based)\cite{wang_survey_2024}, only image-based static FER analysis will be conducted. While dynamic FER methods may capture temporal emotional transitions more effectively, this project centers on evaluating static images to streamline experimentation and ensure consistency.

\item The materials used in this research are limited to licensed and public resources. These include FER datasets, academic publications, source code, programming languages, and libraries, ensuring transparency, reproducibility, and ethical compliance in all aspects of the study.
\end{itemize}

