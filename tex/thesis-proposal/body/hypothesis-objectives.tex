\chapter{Hypothesis and Objectives}
\label{chapter:hypothesis-objectives}
Example text:

This chapter presents the hypothesis formulated for the thesis. Also, it displays the established objectives with its corresponding deliverables. Last section gives a summary of the scope and limitations defined for this research.

\section{Hypothesis}
\label{section:hypothesis}
A brief context description of the formulation of the hypothesis can go here, followed by.
The hypothesis for this research effort reads as follows (example):

%``Our proposed method $P$ in the research can improve by $Q$\footnote{If numbers are involved, usually an explanation of why that specific number was used is given in a footnote \cite{examplereference2}.} the relation in metric 1 and metric 2\footnote{The \cite{examplereference} definition of is used.}.''


''By systematically exploring and evaluating various modifications to the Multi-Layer Perceptron (MLP) head architecture of a Visual Transformer, specifically tailored for Facial Expression Recognition (FER), is posible to identify novel and effective approaches to enhance the model's performance on the RAF-DB dataset. These investigations will contribute to a deeper understanding of the role of the MLP head in Visual Transformers for FER and provide valuable insights for future research and applications.''


\section{Objectives}
\label{section:objectives}
The current research established the following objectives:

\subsection{Main Objective}
\label{section:main-objective}
\label{objective:main}
%Main Objective Text.

To enhance the performance of Visual Transformers in Facial Expression Recognition by introducing novel modifications to the MLP head architecture and evaluating their effectiveness on an In-the-Wild Real World Affective Faces Dataset(RAF-DB).

\subsection{Specific Objectives}
\label{section:specific-objectives}

\begin{enumerate}
	\item Analyze how the proposed modifications to the MLP head architecture influence model transparency and interpretability. Evaluate whether the modified architecture provides more insightful attention maps or feature representations that explain the model’s decision-making process in FER. \todo{check for these attention maps}
	
\item valuate the modified Visual Transformer model to adversarial image environments, noise, or data perturbations (e.g., occlusions, lighting variations, or slight facial deformations) or human fariness such as demographic biases (age, gender, etchnicity). 
	
\item Explore how the new MLP head architectures improve the model’s resilience to a in-the-wild dataset by employing standard FER evaluation metrics (e.g., accuracy, F1-score) as well as metrics that specifically address in-the-wild challenges for example head pose-invariant accuracy or occlusion robustness \todo{check more info for metrics}.
	
\end{enumerate}

\subsection{Deliverables}
Example text:
This section provides the deliverables assigned for each specific objective defined for the thesis.

\textbf{Specific Objective 1:}
Deliverable for specific objective 1. 

\textbf{Specific Objective 2:}
Deliverable for specific objective 2.

\textbf{Specific Objective 3:}
Deliverable for specific objective 3.

\section{Scope and Limitations}

Should include the scope and limitations of the developed research.

\begin{itemize}
\item This research focuses exclusively on predicting 7 discrete class FER emotions, leaving out of the picture other FER class predictions approaches like Action Units or Valence-Arousal \citep{kollias_affect_2021}.

\item The project will not include research that aims to analyze emotional states from dynamic sequences (video-based)\cite{wang_survey_2024}, only image-based static FER analysis will be conducted. 

\item The materials used in this research are limited to licensed and public resources. This includes datasets, research papers, source code, programming language and libraries. 
\end{itemize}

\textbf{Scope and limitation 1:} 

\textbf{Scope and limitation 2:}
 
\textbf{Scope and limitation 3:} 