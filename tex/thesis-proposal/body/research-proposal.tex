\chapter{Research Proposal}
\label{chapter:research-proposal}

\section{Design/Method/Algorithm/Mechanism Proposal}

Explain the proposed design/method/algorithm/mechanism. A diagram is usually beneficial for understanding how the proposal would work.
 
\section{Experiment Design}
\label{section:experiment-design}
Establish the experiment design that will be used to evaluate the hypothesis.
\section{Factors and Levels (if needed):}
Determine and describe the factors and levels that will be used in the experiment.

\section{Measures and Combinations of the Experiment (if needed)}

The example \textbf{Table \ref{tab:table}} summarizes the factors and levels used in the experiments.
\begin{table}[h]
	\caption{Example table.}
	\label{tab:table}
	\begin{center}
		\begin{tabular}{|l|l|l|l|l|}
			\cline{2-5}
			\multicolumn{1}{c|}{}	& \multicolumn{4}{c|}{Factor} \\ 
			\cline{2-5}
			\multicolumn{1}{c|}{}	& Factor 1 & Factor 2 & Factor 3 & Factor 4 \\ \hline
			Levels
			& Level 1 & Level 1	& Level 1  & Level 1   \\
			& Level 2 & Level 2  & Level 2 & Level 2 \\
			& Level 3 & Level 3 & Level 3 & Level 3 \\
			& Level 4 &    & Level 4  &   \\ 
			&  &    & Level 5  &   \\ 
			&  &    & Level 6  &     \\\hline
		\end{tabular}
	\end{center}
\end{table}

Justify the number of runs and repetitions defined for the experiment.


\section{Response Variable}
Description of the response variable used in the research.

\section{Experiment set-up}

\section{Data Recollection}

Description of the data recollection process. 

\section{Statistical Tool Used Description}

Include a section briefly explaining the statistical tool used to analyze the data from the experiments.